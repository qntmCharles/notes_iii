\documentclass{jknotes}
\usepackage{../joshkirklin}
\usepackage{siunitx}
\usepackage{starfont}

\DeclareSIUnit\erg{erg}

\setmathfont{Latin Modern Math}
\setmathfont{GFS NeoHellenic Math}[range=bfsfup/{greek,Greek}->it]
\setmathfont{GFS NeoHellenic Math}[range=sfup/{latin,Latin}->it]

\usetikzlibrary{shapes.misc}

\tikzset{cross/.style={cross out, draw=black, minimum
size=2*(#1-\pgflinewidth), inner sep=0pt, outer sep=0pt}}
%default radius will be 1pt. 

\newcommand{\myol}[2][3]{{}\mkern#1mu\overline{\mkern-#1mu#2}}

\begin{document}

\institution{Cambridge Part III Maths}
\title{Astrophysical Fluid Dynamics}
\lecturer{Dr. Gordon Ogilvie}
\notetaker{Charles Powell}
\date{Lent 2020}

\maketitle
\suggestionsspiel
\tableofcontents

\lecture{22/01/21}
\section{Introduction}
\subsection{Areas of application}
Astrophysical fluid dynamics (AFD) is relevant to the description of the interiors
of stars and planets, exterior phenomena such as discs, winds and jets, the
interstellar medium, the intergalactic medium, and cosmology itself. A fluid
description is not applicable in regions that are solidified, such as the
rocky or icy cores of giant planets and the crusts of neutron stars, and also
in very genuous regions where the medium is not sufficiently collisional.

\subsection{Theoretical varieties}
Various flavours of AFD are in use. The basic models we will consider are:
\paragraph{Hydrodynamics (HD) / Newtonian gas dynamics:}
This model is non-relativistic, compressible, ideal (inviscid and adiabatic),
self-gravitating, and usually assumes a perfect gas.
\paragraph{Magnetohydrodynamics (MHD):}
This model is the same as above, with the addition of a magnetic field. We
will often use ideal MHD, which assumes a perfectly conducting fluid.

\subsection{Characteristic features}
The elements of theory often important in AFD are compressibility,
gravitation, and thermal physics. Sometimes, magnetic fields, radiation
forces, and relativity are important. Rarely important aspects are viscosity,
surface tension, and solid boundaries.

\subsection{Useful data}
Some useful data for the course, in CGS (centimetres, grams, seconds) units:
\begin{table}[h]
\begin{tabular}{ll}
	Newton's constant & $G = 6.674 \times 10^{-8} \,\si{\cm^3.\g^{-1}.\s^{-2}}$ \\
	Boltzmann's constant & $k = 1.381 \times 10^{-16} \, \si{\erg.\K^{-1}}$ \\
	Stefan's constant & $\sigma = 5.670 \times 10^{-5}
	\,\si{\erg.\cm^{-2}.s^{-1}.K^{-4}}$\\
	Speed of light & $c = 2.998 \times 10^{1-} \,\si{\cm.s^{-1}}$\\
	Hydrogen mass & $m_H = 1.674 \times 10^{-24} \, \si{\g}$\\
	Solar mass & $M_\Sun = 1.988 \times 10^{33} \,\si{\g}$\\
	Solar radius & $R_\Sun = 6.957 \times 10^{10} \, \si{\cm}$\\
	Solar luminosity & $L_\Sun = 3.828 \times 10^{33} \, \si{\erg \s^{-1}}$\\
	Parsec & $pc = 3.086 \times 10^{18} \, \si{\cm}$\\
	Astronomial unit (AU) & $au = 1.496 \times 10^{13} \, \si{\cm}$\\
	Joule erg conversion & $1 \si{\J} = 10^7 \si{\erg}$
\end{tabular}
\end{table}

\section{Ideal gas dynamics}
\subsection{Fluid variables}
A fluid is characterised by a velocity field $\symbf{u}(\symbf{x},t)$ and two
independent thermodynamic properties. Most useful are the dynamical variables:
the pressure $p(\symbf{x},t)$ and the mass density $\rho(\symbf{x},t)$. Other
properties, e.g. temperature $T$, can be regarded as functions of $p$ and
$\rho$. The \emph{specific volume} (volume per unit mass) is $v = 1/\rho$. 

We neglect the possible complications of variable chemical composition
associated with chemical and nuclear reactions, ionisation and recombination.

\subsection{Eulerian and Lagrangian viewpoints}
In the \emph{Eulerian} viewpoint we consider how fluid properties vary in time
at a point which is fixed in space, i.e. attached to the (usually inertial)
coordinate system. The Eulerian time derivative is simply $\partial_t$.

In the \emph{Lagrangian} viewpoint we consider how fluid properties vary in
time at a point which moves with the fluid at velocity
$\symbf{u}(\symbf{x},t)$. The Lagrangian time derivative (or material
derivative) is
\begin{equation}
	\frac{\diffD}{\diffD t} = \frac{\partial}{\partial t} +
	\symbf{u}\cdot\nabla
\end{equation}

\subsection{Material points and structures}
A material point is an idealised fluid element, a point that moves with the
bulk velocity $\symbf{u}(\symbf{x},t)$ of the fluid. Note that the true
particles of which the fluid is composed also have random thermal motion.

Material curves, surfaces and volumes are geometrical structures composed of
fluid elements; they move with the fluid flow and are deformed by it. An
infinitesimal material line element $\delta\symbf{x}$ evolves according to
\begin{equation}
	\frac{\diffD \delta \symbf{x}}{\diffD t} = \delta \symbf{u} =
	\delta\symbf{x}\cdot\nabla\symbf{u}
\end{equation}
It changes its length and/or orientation in the presence of a velocity
gradient.

Infinitesimal material surface or volume elements can be defined from two or
three material line elements according to the vector product and the triple
scalar product.
\begin{align}
	\delta \symbf{S} &= \delta \symbf{x}^{(1)} \times \delta \symbf{x}^{(2)} \\
	\delta V &= \delta \symbf{x}^{(1)} \cdot \delta \symbf{x}^{(2)} \times
	\delta \symbf{x}^{(3)}
\end{align}
They evolve according to
\begin{align}
	\frac{\diffD \delta \symbf{S}}{\diffD t} &= (\nabla \cdot \symbf{u})\delta
	\symbf{S} - \nabla \symbf{u} \cdot\delta \symbf{S}\\
	\frac{\diffD \delta V}{\diffD t} &= (\nabla \cdot \symbf{u})\delta V
\end{align}
The second result is easier to understand: the volume element increases when
the flow is divergent.

\subsection{Equation of mass conservation}
The equation of mass conservation
\begin{equation}
	\frac{\partial \rho}{\partial t} + \nabla \cdot (\rho \symbf{u}) = 0
\end{equation}
has typical form of conservation law: rate of change of a density and
divergence of a flux. Here, $\rho$ is mass density and $\rho \symbf{u}$ is mass
flux density.  An alternative form of the same equation is
\begin{equation}
	\frac{\diffD \rho}{\diffD t} = - \rho \nabla \cdot \symbf{u}
\end{equation}
If $\delta m = \rho \delta V$ is a material mass element, it can be seen that
mass is conserved in the form
\begin{equation}
	\frac{\diffD \delta m }{\diffD t} = 0
\end{equation}

\subsection{Equation of motion}
The equation of motion
\begin{equation}
	\rho \frac{\diffD \symbf{u}}{\diffD t} = - \rho \nabla \Phi - \nabla p
\end{equation}
derives from Newton's second law per unit volume with gravitational and
pressure forces. The gravitational potential is $\Phi(\x,t)$ and $\symbf{g} =
-\nabla\Phi$ is the gravitational field.

The force due to pressure acting on a volume $V$ with bounding surface $S$ is
\begin{equation}
	-\int_S p\, \diffd \symbf{S} = \int_V (-\nabla p) \, \diffd V
\end{equation}
Viscous forces are neglected in ideal gas dynamics.

\subsection{Poisson's equation}
The gravitational potential is related to the mass density by Poisson's
equation
\begin{equation}
\nabla^2 \Phi = 4\pi G \rho
\end{equation}
where $G$ is Newton's constant. The solution
\begin{equation}
	\Phi = \Phi_{\text{int}} + \Phi_{\text{ext}} = - G \int_V
	\frac{\rho(\x',t)}{\abs{\x'-\x}} \, \diffd^3 \x' - G \int_{\hat{V}} 
	\frac{\rho(\x',t)}{\abs{\x'-\x}} \, \diffd^3 \x' 
\end{equation}
generally involves contributions form both the fluid region $V$ under
consideration and the exterior region $\hat{V}$. A \emph{non-self-gravitating}
fluid is one of negligible mass for which $\Phi_{\text{int}}$ can be
neglected. More generally, the \emph{Cowling approximation} consists of
treating $\Phi$ as being specified in advance, so that Poisson's equation is
not coupled to the other equations.

\subsection{Thermal energy equation and equation of state}
In the absence of non-adiabatic heating (e.g. by viscous dissipation or
nuclear reactions) and cooling (e.g. by radiation or conduction),
\begin{equation}
	\frac{\diffD s}{\diffD t} = 0
\end{equation}
where $s$ is the \emph{specific entropy} (entropy per unit mass). Fluid
element undergo reversible thermodynamic changes and preserve their entropy
(adiabatic flow). This condition is violated in shocks (see section 6).

The thermal variables $(T,s)$ can be related to the dynamical variables
$(P,\rho)$ via an \emph{equation of state} and standard thermodynamic
identities. The most important case is that of an \emph{ideal gas} with
\emph{blackbody radiation}
\begin{equation}
	p = p_g + p_r = \frac{k \rho T}{\mu m_H} + \frac{4\sigma T^4}{3c}
\end{equation}
where $k$ is Boltzmann's constant, $m_H$ is mass of a hydrogen atom, $\sigma$
is Stefan's constant, $c$ is the speed of light, and $\mu$ is the mean
molecular weight, defined as the average mass of the particles in units of
$m_H$, equal to
\begin{itemize}
	\item $2.0$ for molecular hydrogen
	\item $1.0$ for atomic hydrogen
	\item $0.5$ for fully ionised hydrogen
	\item about $0.6$ for ionised matter of typical cosmic abundances.
\end{itemize}
The component $p_g$ is the \emph{gas pressure} and $p_r$ is the
\emph{radiation pressure}. Radiation pressure is usually negligible except in
the centres of high mass stars and in the immediate environments of neutron
stars and black holes. The pressure of an ideal gas is often written in the
form $\mathcal{R}\rho T/\mu$ where $\mathcal{R} = k/m_H$ is a version of the
universal gas constant.

We define the \emph{first adiabatic exponent}
\begin{equation}
	\Gamma_1 = \left( \frac{\partial \log p}{\partial \log \rho}\right)_S
\end{equation}
which is related to the ratio of specific heat capacities
\begin{equation}
	\gamma = \frac{c_p}{c_v} = \frac{T \left(\frac{\partial s}{\partial
	T}\right)_p}{T \left( \frac{\partial s}{\partial T}\right)_V}
\end{equation}
by $\Gamma_1 = \chi_\rho \gamma$ where
\begin{equation}
	\chi_\rho = \left( \frac{\partial \log p}{\partial \log \rho}\right)_T
\end{equation}
can be found from the equation of state. We can then rewrite the thermal
energy equation as
\begin{equation}
	\frac{\diffD p}{\diffD t} = \frac{\Gamma_1 p}{\rho} \frac{\diffD
	\rho}{\diffD t} = - \Gamma_1 p \nabla \cdot \symbf{u}
\end{equation}

For an ideal gas with negligible radiation pressure, $\chi_\rho =1$ and so
$\Gamma_1 = \gamma$. Adopting this very common assumption, we write
\begin{equation}
	\frac{\diffD p}{\diffD t} = -\gamma p \nabla \cdot \symbf{u}
\end{equation}

\subsection{Simplified models}
A \emph{perfect gas} may be defined as an ideal gas with constant $c_v, c_p,
\gamma$ and $\mu$. Equipartition of energy for a classical gas with $n$
degrees of freeodm per particle gives
\begin{equation}
	\gamma = 1 + \frac{2}{n}
\end{equation}

For a classical monatomic gas with $n=3$ translational degrees of freedom,
$\gamma = 5/3$. This is relevant for fully ionised matter. For a classical
diatomic gas with two addditional rotational degrees of freeodm, $n=5$ and
$\gamma = 7/5$. This is relevant for molecular hydrogen. In reality,
$\Gamma_1$ is variable when the gas undergoes ionisation or when the gas and
radiation pressures are comparable. 

The specific \emph{internal energy} (or \emph{thermal energy}) of a perfect
gas is
\begin{equation}
	e = \frac{p}{(\gamma-1)\rho} = \frac{n}{\mu m_H} \frac{1}{2}kT
\end{equation}
Note that each particle has an internal energy of $\frac{1}{2}kT$ per degree
of freedom, and the number of particles per unit mass is $1/{\mu m_H}$.

A \emph{barotropic} fluid is an idealised situation in which the relation $p(\rho)$
is known in advance. We can then dispense with the thermal energy equation.
For example, if the gas is strictly isothermal and perfect, $p = c_s^2 \rho$
with the constant $c_s$ being the isothermal sound speed. Alternatively, if
the gas is strictly homentropic (constant $s$) and perfect, then $p = K
\rho^\gamma$ with $K$ constant. 

An \emph{incompressible} fluid is an idealised situation in which
$\frac{\diffD \rho}{\diffD t} = 0$, implying $\nabla \cdot \symbf{u} = 0$.
This can be achieved formally by taking the limit $\gamma \to \infty$. The
approximation of incompressibility eliminates acoustic phenomena from the
dynamics. The ideal gas law itself is not valid at very high densities or
where quantum degeneracy is important.

\subsection{Microphysical basis}
It is useful to understand the way in which the fluid-dynamical equations are
derived microphysical considerations. The simplest model involves identical
neutral particles of mass $m$ of negligible size with no internal degrees of
freedom.

\subsubsection{Boltzmann equation}
Between collisions, particles follow Hamiltonian trajectories in their six
dimensional $(\x, \symbf{v})$ phase space:
\begin{equation}
	\dot{x}_i = v_i, \hspace{2em} \dot{v}_i = a_i - \frac{\partial
	\Phi}{\partial x_i}
\end{equation}

The distribution funnction $f(\x, \symbf{v},t)$ specifies the number density
of particles in phase space. The velocity moments of $f$ define the number
$n(\x,t)$ in real space, the bulk velocity $\symbf{u}(\x,t)$ and the velocity
dispersion $c(\x,t)$ according to
\begin{align}
	\int f \, \diffd^3 \symbf{v} &= n\\
	\int \symbf{v} f \, \diffd^3 \symbf{v} &= n \symbf{u} \\
	\int \abs{\symbf{v}-\symbf{u}}^2 f \, \diffd^3 \symbf{v} &= 3nc^2
\end{align}
Equivalently,
\begin{equation}
	\int v^2 f \, \diffd^3 \symbf{v} = n (\symbf{u}^2 + 3c^2)
\end{equation}
The relation between velocity dispersion and temperature is $kT = mc^2$. In
the absence of collisions, $f$ is conserved following the Hamiltonian flow in
phase space. This is because particles are conserved and the flow in phase
space is incompressible. More generally, $f$ evolves according to
\emph{Boltzmann's equation}
\begin{equation}
	\frac{\partial f}{\partial t} + v_j \frac{\partial f}{\partial x_j} + a_j
	\frac{\partial f}{\partial v_j} = \left( \frac{\partial f}{\partial
	t}\right)_c
\end{equation}

The collision term on the RHS is a complicated integral operator but has 3
simple properties corresponding to the conservation of mass, momentum and
energy in collisions.
\begin{align}
	\int m \left( \frac{\partial f}{\partial t}\right)_c \, \diffd^3 \symbf{v}
	&= 0 \\
	\int m \symbf{v}\left( \frac{\partial f}{\partial t}\right)_c \, \diffd^3 \symbf{v}
	&= 0 \\
	\int \frac{1}{2}m\symbf{v}^2 \left( \frac{\partial f}{\partial t}\right)_c \, \diffd^3 \symbf{v}
	&= 0
\end{align}
The collision term is local in $\x$ (not even involving derivatives) although
it does involve integrals over $\symbf{v}$. The equation $\left(\frac{\partial
f}{\partial t}\right)_c = 0$ has the general solution
\begin{equation}
	f = f_M = (2\pi c^2)^{-3/2} n \exp\left( -
	\frac{\abs{\symbf{v}-\symbf{u}}^2}{2c^2}\right)
\end{equation}
with parameters $n, \symbf{u}$ and $c$ that may depend on $\x$. This is the
\emph{Maxwellian distribution}. 

\subsubsection{Derivation of fluid equations}
A crude but illuminating model of the collision operator is the \emph{BGK
approximation}
\begin{equation}
	\left( \frac{\partial f}{\partial t}\right)_c \approx - \frac{1}{\tau}
	(f-f_M)
\end{equation}
where $f_M$ is a Maxwellian distribution with the same $n, \symbf{u}$ and $c$
as $f$, and $\tau$ is the \emph{relaxation time}. This timescale $\tau$ can be
identified approximately with the mean flight time of particles between
collisions. The collisions attempt to restore a Maxwellian distribution on a
characteristic timescale $\tau$. They do this by randomising the particle
velocities in a way consistent with the conservation of momentum and energy.
If the characteristic timescale of the fluid flow is much greater than $\tau$,
then the collision term dominates the Boltzmann equation and $f$ is very close
to $f_m$. This is the \emph{hydrodynamic limit}. 

The velocity moments of $f_M$ can be determined from standard Gaussian
integrals, in particular
\begin{align}
	\int f_M \, \diffd^3 \symbf{v} &= n \\
	\int v_i f_M \, \diffd^3 \symbf{v} &= n u_i \\
	\int v_i v_j f_M \, \diffd^3 \symbf{v} &= n(u_i u_j + c^2 \delta_{ij}) \\
	\int \symbf{v}^2 v_i f_M \, \diffd^3 \symbf{v} &= n(\symbf{u}^2 + 5 c^2)
	u_i
\end{align}
We obtain equations for mass, momentum and energy by taking moments of the
Boltzmann equation weighted by $(m, mv_i, \frac{1}{2}m\symbf{v}^2)$. In each
case the collision term integrates to $0$ because of its conservative
properties, and the $\partial/\partial v_j$ term can be integrated by parts.
We replace $f$ with $f_M$ when evaluating the LHS and note that $mn = \rho$:
\begin{align}
	\frac{\partial \rho}{\partial t} + \frac{\partial}{\partial x_i} (\rho u_i) &= 0 \\
	\frac{\partial}{\partial t}(\rho u_i) + \frac{\partial}{\partial x_j}
	\left[ \rho (u_i u_j + c^2 \delta_{ij})\right] - \rho a_i &= 0 \\
	\frac{\partial}{\partial t} \left[\frac{1}{2}\rho \symbf{u}^2 + \frac{3}{2}\rho
	c^2\right] + 
	\frac{\partial}{\partial x_i} \left[ (\frac{1}{2}\rho \symbf{u}^2 +
	\frac{5}{2}\rho c^2) u_i\right] - \rho u_i a_i &= 0
\end{align}

These are equivalent to the equations of ideal gas dynamics in conservative
form (see section 4) for a monatomic ideal gas ($\gamma = 5/3$). The specific
internal energy is $\rho = \frac{3}{2}c^2 = \frac{3}{2}\frac{kT}{m}$.



\end{document}
