\documentclass{jknotes}
\usepackage{../joshkirklin}
\usepackage{siunitx}
\usepackage{starfont}

\DeclareSIUnit\erg{erg}

\setmathfont{Latin Modern Math}
\setmathfont{GFS NeoHellenic Math}[range=bfsfup/{greek,Greek}->it]
\setmathfont{GFS NeoHellenic Math}[range=sfup/{latin,Latin}->it]

\usetikzlibrary{shapes.misc}

\tikzset{cross/.style={cross out, draw=black, minimum
size=2*(#1-\pgflinewidth), inner sep=0pt, outer sep=0pt}}
%default radius will be 1pt. 

\newcommand{\myol}[2][3]{{}\mkern#1mu\overline{\mkern-#1mu#2}}

\begin{document}

\institution{Cambridge Part III Maths}
\title{Astrophysical Fluid Dynamics}
\lecturer{Dr. Gordon Ogilvie}
\notetaker{Charles Powell}
\date{Lent 2020}

\maketitle
\suggestionsspiel
\tableofcontents

\lecture{22/01/21}
\section{Introduction}
\subsection{Areas of application}
Astrophysical fluid dynamics (AFD) is relevant to the description of the interiors
of stars and planets, exterior phenomena such as discs, winds and jets, the
interstellar medium, the intergalactic medium, and cosmology itself. A fluid
description is not applicable in regions that are solidified, such as the
rocky or icy cores of giant planets and the crusts of neutron stars, and also
in very genuous regions where the medium is not sufficiently collisional.

\subsection{Theoretical varieties}
Various flavours of AFD are in use. The basic models we will consider are:
\paragraph{Hydrodynamics (HD) / Newtonian gas dynamics:}
This model is non-relativistic, compressible, ideal (inviscid and adiabatic),
self-gravitating, and usually assumes a perfect gas.
\paragraph{Magnetohydrodynamics (MHD):}
This model is the same as above, with the addition of a magnetic field. We
will often use ideal MHD, which assumes a perfectly conducting fluid.

\subsection{Characteristic features}
The elements of theory often important in AFD are compressibility,
gravitation, and thermal physics. Sometimes, magnetic fields, radiation
forces, and relativity are important. Rarely important aspects are viscosity,
surface tension, and solid boundaries.

\subsection{Useful data}
Some useful data for the course, in CGS (centimetres, grams, seconds) units:
\begin{table}[h]
\begin{tabular}{ll}
	Newton's constant & $G = 6.674 \times 10^{-8} \,\si{\cm^3.\g^{-1}.\s^{-2}}$ \\
	Boltzmann's constant & $k = 1.381 \times 10^{-16} \, \si{\erg.\K^{-1}}$ \\
	Stefan's constant & $\sigma = 5.670 \times 10^{-5}
	\,\si{\erg.\cm^{-2}.s^{-1}.K^{-4}}$\\
	Speed of light & $c = 2.998 \times 10^{1-} \,\si{\cm.s^{-1}}$\\
	Hydrogen mass & $m_H = 1.674 \times 10^{-24} \, \si{\g}$\\
	Solar mass & $M_\Sun = 1.988 \times 10^{33} \,\si{\g}$\\
	Solar radius & $R_\Sun = 6.957 \times 10^{10} \, \si{\cm}$\\
	Solar luminosity & $L_\Sun = 3.828 \times 10^{33} \, \si{\erg \s^{-1}}$\\
	Parsec & $pc = 3.086 \times 10^{18} \, \si{\cm}$\\
	Astronomial unit (AU) & $au = 1.496 \times 10^{13} \, \si{\cm}$\\
	Joule erg conversion & $1 \si{\J} = 10^7 \si{\erg}$
\end{tabular}
\end{table}

\section{Ideal gas dynamics}
\subsection{Fluid variables}
A fluid is characterised by a velocity field $\symbf{u}(\symbf{x},t)$ and two
independent thermodynamic properties. Most useful are the dynamical variables:
the pressure $p(\symbf{x},t)$ and the mass density $\rho(\symbf{x},t)$. Other
properties, e.g. temperature $T$, can be regarded as functions of $p$ and
$\rho$. The \emph{specific volume} (volume per unit mass) is $v = 1/\rho$. 

We neglect the possible complications of variable chemical composition
associated with chemical and nuclear reactions, ionisation and recombination.

\subsection{Eulerian and Lagrangian viewpoints}
In the \emph{Eulerian} viewpoint we consider how fluid properties vary in time
at a point which is fixed in space, i.e. attached to the (usually inertial)
coordinate system. The Eulerian time derivative is simply $\partial_t$.

In the \emph{Lagrangian} viewpoint we consider how fluid properties vary in
time at a point which moves with the fluid at velocity
$\symbf{u}(\symbf{x},t)$. The Lagrangian time derivative (or material
derivative) is
\begin{equation}
	\frac{\diffD}{\diffD t} = \frac{\partial}{\partial t} +
	\symbf{u}\cdot\nabla
\end{equation}

\subsection{Material points and structures}
A material point is an idealised fluid element, a point that moves with the
bulk velocity $\symbf{u}(\symbf{x},t)$ of the fluid. Note that the true
particles of which the fluid is composed also have random thermal motion.

Material curves, surfaces and volumes are geometrical structures composed of
fluid elements; they move with the fluid flow and are deformed by it. An
infinitesimal material line element $\delta\symbf{x}$ evolves according to
\begin{equation}
	\frac{\diffD \delta \symbf{x}}{\diffD t} = \delta \symbf{u} =
	\delta\symbf{x}\cdot\nabla\symbf{u}
\end{equation}
It changes its length and/or orientation in the presence of a velocity
gradient.

Infinitesimal material surface or volume elements can be defined from two or
three material line elements according to the vector product and the triple
scalar product.
\begin{align}
	\delta \symbf{S} &= \delta \symbf{x}^{(1)} \times \delta \symbf{x}^{(2)} \\
	\delta V &= \delta \symbf{x}^{(1)} \cdot \delta \symbf{x}^{(2)} \times
	\delta \symbf{x}^{(3)}
\end{align}
They evolve according to
\begin{align}
	\frac{\diffD \delta \symbf{S}}{\diffD t} &= (\nabla \cdot \symbf{u})\delta
	\symbf{S} - \nabla \symbf{u} \cdot\delta \symbf{S}\\
	\frac{\diffD \delta V}{\diffD t} &= (\nabla \cdot \symbf{u})\delta V
\end{align}
The second result is easier to understand: the volume element increases when
the flow is divergent.



\end{document}
